%% Title
\pagenumbering{arabic}
\titlepage[\normalsize{
High Energy Physics\\
Blackett Laboratory\\ 
Imperial College London}]%
{A dissertation submitted to Imperial College London\\
  for the degree of Doctor of Philosophy}



%% Abstract
\begin{abstract}

  A novel search for supersymmetric particles in events with one high transverse momentum jet and large missing transverse energy is performed using an integrated luminosity of 19.7~\fbinv of pp collision data collected using the CMS detector at the Large Hadron Collider.
  By using events with an energetic radiated jet, sensitivity to supersymmetric models with compressed mass spectra is gained where the decay products have very low energy.
  Standard Model background estimates are evaluated with the use of data control samples. 
  No excess over Standard Model expectations is observed, and limits are placed on third generation squark production at the 95\% confidence level using supersymmetric simplified models.

  The development of a Level 1 trigger algorithm to reconstruct jets in the Phase 1 Upgrade of the CMS detector is presented. 
  Utilising the full granularity of the CMS calorimeter and time-multiplexed-trigger technology, a new algorithm with increased flexibility and resolution is presented. 
  It is possible to measure and subtract the contribution to calorimeter deposits of soft particles originating from multiple pp vertices in an event, on an event-by-event basis, in order to decrease the trigger rates associated with high luminosity future run conditions. 
  This will enable CMS to maintain, or better, its ability to do physics as centre-of-mass energy and instantaneous luminosity of the LHC increases in the years to come.

\end{abstract}


%% Declaration
\begin{declaration}
  I, the author of this thesis, hereby declare the material presented here 
  to be the result of my own work, except where explicit
  reference is made to the work of others.
  It has not been submitted for another qualification to this or any other university. 
  All figures labelled ``CMS'' have been sourced from CMS publications, referenced in the caption, and include those produced by the author.
  Those figures labelled ``CMS Preliminary'' have been sourced from a CMS public preliminary document or an unpublished CMS document.
  Figures labelled ``CMS Simulation'' are made using CMS simulation only, and those labelled ``CMS Unpublished'' do not feature in any CMS documents.
  All figures taken from external sources are referenced appropriately throughout this thesis.
  \vspace*{1cm}
  \begin{flushright}
   Robyn Lucas 
  \end{flushright}
{\it The copyright of this thesis rests with the author and is made available under a Creative Commons Attribution Non-Commercial No Derivatives licence. Researchers are free to copy, distribute or transmit the thesis on the condition that they attribute it, that they do not use it for commercial purposes and that they do not alter, transform or build upon it. For any reuse or redistribution, researchers must make clear to others the license terms of this work.}
\end{declaration}


%% Acknowledgements
\begin{acknowledgements}
  Firstly, I would like to thank my parents and family for their support and encouragement throughout my academic career.
  Secondly, I thank my supervisors Alex Tapper and Steve Worm for their advice and guidance.
  I also express huge thanks to the Monojet analysis team; without Phat Srimanobhas my research would not have been possible, and whose patience, help and hard work I am eternally grateful for; Anwar Bhatti, Sarah Malik, Shuichi Kunori and Teruki Kamon, whose discussions and guidance have been invaluable.
  Thanks also to Michele Pioppi and Andrew Rose for their patience in helping a fresh faced first year who could not code in the work on upgrade jets.
  My friends and colleagues; Sam Cunliffe, Darren Burton, Andrew Gilbert, Matthew Kenzie and Patrick Owen you provided much needed distractions and laughs, as well as more than a little bit of help over the years. 
  Sam Hall must get my particular thanks for the many occasions where he fixed my code and made me dinner!
  I cannot express enough thanks to my boyfriend Ed. 
  His unfaltering support, both in Geneva and London, were at times the only thing keeping me going. Sorry I never found the Lucas boson!
  Finally, I thank the STFC for providing the funding that enabled me to conduct the research presented here, and for the 15 months spent out at CERN.
\end{acknowledgements}


% %% Preface
% \begin{preface}
%   This thesis describes my research on various aspects of the \CMS
%   particle physics program, centred around the \CMS detector and \LHC
%   accelerator at \CERN in Geneva.

%   \noindent
%   For this example, I'll just mention \ChapterRef{chap:SomeStuff}
%   and \ChapterRef{chap:MoreStuff}.
% \end{preface}

%% ToC
\tableofcontents

%% Strictly optional!
\frontquote%
  {For Peter}%
  {}
