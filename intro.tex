\chapter{Introduction}
\label{chap:intro}

%\chapterquote{I may not be there yet, but I am closer than I was yesterday}
%{Unknown}

%% Note that the citations in this chapter use the journal and 
%% arXiv keys: I used the SLAC-SPIRES online BibTeX retriever 
%% to build my bibliography. There are also quite a few non-standard
%% macros, which come from my personal collection. You can have them
%% if you want, or I might get round to properly releasing them at 
%% some point myself.

\chapterquote{Everything starts somewhere, although many physicists disagree.}
{Terry Pratchett, 1948--2015}

The \ac{SM} of particle physics, a theory developed during the second half of the 20$^{\rm th}$ Century, represents mankind's best understanding of the universe around us: what we are made of and the interactions in the world around us. 
A multitude of theoretical leaps and experimental discoveries have cemented it as one of the cornerstones of physics. 
It has withstood unparalleled experimental scrutiny~\cite{PDG} and is incredibly successful at explaining the vast majority of experimental phenomena we observe.

However, the \ac{SM} fails to explain some fairly crucial aspects of the universe that we observe on both a daily basis and through more in depth inspections. 
Gravity has no place within the \ac{SM}; it is not one of the fundamental forces of nature that it describes. 
The presence of any matter at all in the universe, and lack of any antimatter raises a crucial question: why did everything not annihilate in the instants after the big bang?
Cosmological observations tell us that \ac{DM} accounts for 27\% of the \replaced{observable}{visible} universe - an additional, non-luminous but gravitating matter must account for the extra gravity seen in observations of gravitational lensing, galaxy rotation curves, and the \ac{CMB}.
Yet, there is no candidate for such a particle in the \ac{SM}. 
Questions which are more theoretical in nature persist - why are there four forces, and why do they have such different strengths? 
And why do we observe three generations of particles? Further, why is the third generation's top quark, for example, so much heavier than the other quarks, and why is the neutrino's mass so small? In fact, why do neutrinos have a mass at all?

The \ac{SM}, then, is thought to be some low-energy approximation of a more complete picture.
One widely considered theory, which goes some way to solving many of the issues with the \ac{SM}, is \ac{SUSY}.
It introduces a new symmetry of nature between fermions and bosons, postulating a new partner sparticle for every \ac{SM} particle. 
In common forms of the theory, in which a sparticle will always decay into another sparticle, the \ac{LSP} provides a \ac{DM} candidate. 
It can enable the weak, strong and electromagnetic forces to unite, and provide a solution to the problem of quadratic divergences in the Higgs boson's mass. 
A detailed description of the theory of the \ac{SM}, the motivation for new physics, and \ac{SUSY}, is given in Chapter~\ref{chap:theory}.

The LHC was built to probe our understanding of fundamental physics. 
Its primary aim, to discover the Higgs boson, the last remaining piece of the \ac{SM}, was a resounding success with the discovery \added{of a neutral boson} from both the \ac{ATLAS} and \ac{CMS} collaborations unveiled to the world on 4$^{\rm th}$ July \replaced{2012}{2014}~\cite{Aad:2012tfa,Chatrchyan:2012ufa}, \added{which, with further observations, looks to be compatible with the \ac{SM} Higgs boson.}  
\added{As well as Higgs exploitation,} attention has now shifted to \ac{BSM} physics, including the search for \ac{SUSY} and \ac{DM}. 
A description of the LHC apparatus and the \ac{CMS} detector is given in Chapter~\ref{chap:detector}.

In Chapter~\ref{chap:theory} we discuss the arguments for a particular type of \ac{SUSY} where the various sparticles are close in mass to one other: the mass spectra of the models are ``compressed''. 
Sparticle decay products in these scenarios are typically very low energy as most of the mass-energy of the parent sparticle is taken in creating the daughter sparticle, leaving very little phase space for the accompanying \ac{SM} decay products.
Traditional \ac{SUSY} searches are insensitive to these scenarios because of these soft, \ac{SM} decay products are obscured by the \ac{SM} backgrounds, largely in multijet production. 

Chapter~\ref{chap:sus13009} describes a search dedicated to looking for compressed \ac{SUSY}, using a novel method of looking for events with a monojet topology; events with a high transverse momentum jet which is balanced by large missing transverse momentum. 
The idea is to trigger on compressed \ac{SUSY} events by looking for objects created in association with sparticles; i.e. radiated jets in the initial state, and ignoring the soft sparticle decay products.
The search is therefore independent of the visible decay products, and therefore \added{is able to probe the phase space in which the sparticle in question (here the \sTop{} and \sBot{} squarks) are mass-degenerate with the \ac{LSP}.} 
The event selection is optimised to search for compressed scenarios and backgrounds to the search are estimated using data driven methods alongside simulation.
Chapter~\ref{chap:sus13009results} shows the results of the search. No excess above the \ac{SM} expectations is observed, so limits are set on models of compressed \ac{SUSY} in the third generation.


The author was responsible for the search described in Chapters~\ref{chap:sus13009} and~\ref{chap:sus13009results} as a part of the CMS monojet group.
This search has been released as a CMS Physics Analysis Summary, see Ref.~\cite{sus13009}. 
It has very recently been combined with two other CMS \ac{SUSY} analyses into \added{a} legacy Run I paper~\cite{sus14001} on third generation squark production in all\added{-}hadronic final states. 
The author was the main editor of this paper and it has just been \replaced{published}{submitted} in \ac{JHEP}.
\added{The search was interpreted in the context of supersymmetry, however could be re-interpreted in many other new physics models which contain a boosted system, with an \ac{ISR} jet and low energy decay products, alongside a large imbalance in transverse momentum.}




The LHC was built to be at the frontier of fundamental physics for decades to come. 
A programme of upgrades is planned well into the 2020's, which will continue to extend the reach of the accelerator and detectors in terms of both energy and integrated luminosity. 
As well as increasing the energy up to 13~\TeV, and eventually to the design energy of 14~\TeV, the LHC will deliver proton beams at increasingly high instantaneous luminosities, to enable precision measurements of very rare processes such as the properties of the newly discovered boson. 
The CMS detector must be prepared to cope with the enormous challenges that such increases in the instantaneous luminosity will bring: many overlapping pp vertices in each bunch crossing lead to much higher detector occupancies and many more events per second. 
The trigger system, used to filter out the events of interest from the vastly more numerous `uninteresting' events, will therefore have to cope with huge increases in rates while maintaining sensitivity to new physics processes. 
The CMS \ac{L1} system, the first stage of this sieve, is therefore undergoing an upgrade. 
Chapter~\ref{chap:l1jets} details the development of a new jet algorithm for the \ac{L1} trigger upgrade that the author conducted during 2012--2013, which as well as being far more flexible than the current algorithm, has event-by-event \ac{PU} subtraction and shows a reduction in rates for hadronic triggers.





